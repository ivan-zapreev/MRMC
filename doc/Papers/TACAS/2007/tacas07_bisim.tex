
\documentclass{llncs}

\usepackage{amssymb}
\usepackage{amsmath}
\usepackage{latexsym}
\usepackage{epsf}
%\usepackage{psfig}
\usepackage{epsfig}
\usepackage{graphicx}
\usepackage{arrows}
\usepackage{color}
\usepackage{ifthen}
\usepackage{wrapfig}


\newcommand{\Prob}[3]{{\cal P}_{#1 #2}(#3)}
\newcommand{\U}{\mathrel{\mathcal{U}}}
\newcommand{\W}{\mathrel{\mathcal{W}}}
\newcommand{\rU}[1]{\mathrel{\mathcal{U}^{#1}}}

\newcommand{\F}{\Diamond}

\newcommand{\mat}[1]{{\bf #1}}
\newcommand{\vt}[1]{\underline{#1}}
\newcommand{\sz}[1]{{\rm size}({\bf #1})}
\newcommand{\diag}[1]{\mbox{\sl diag}(#1)}
\newcommand{\nats}{\mathbb{N}}

\def\Ende{\mbox{\rule{6pt}{6pt}}}
\newcommand{\RateMatrix}{\mathbf{R}}
\newcommand{\ExitRate}{E}
\newcommand{\TransProb}{\mathbf{P}}
\newcommand{\M}{\mathcal{C}}
\newcommand{\AP}{\mbox{\sl AP}}
\newcommand{\Path}{\mbox{\sl Path}}
\newcommand{\Reach}{\mbox{\sl Reach}}
\newcommand{\Cylinder}{\mbox{\sl C}}
\newcommand{\Real}{\mathbb{R}}
\newcommand{\duration}{\delta}
\newcommand{\set}[1]{\{ \, #1 \, \} }
\newcommand{\true}{\mathrm{tt}}
\newcommand{\false}{\mathrm{ff}}
\newcommand{\Sat}{\mbox{\sl Sat}}
\newcommand{\smSat}{\mbox{\scriptsize \sl Sat}}
\newcommand{\C}{\mathcal{C}}
\newcommand{\D}{\mathcal{D}}
\renewcommand{\F}{F}
\newcommand{\bfQ}{\mathbf{R}}

\newcommand{\emb}{\mbox{\sl emb}}
\newcommand{\smemb}{\mbox{\scriptsize \sl emb}}
\newcommand{\unif}{\mbox{\sl unif}\,}
\newcommand{\smunif}{\mbox{\scriptsize \sl unif}}

\newcommand{\rew}{r}
%% \newcommand{\text}[1]{\mbox{#1}}

\newcommand{\cslequiv}{\equiv_{\text{\tiny CSL}}}
\newcommand{\cslnoxequiv}{\equiv_{\text{\tiny CSL}_{{\setminus}X}}}
\newcommand{\pctlequiv}{\equiv_{\text{\tiny PCTL}}}
\newcommand{\pctlnoxequiv}{\equiv_{\text{\tiny PCTL}_{{\setminus}X}}}
\newcommand{\simpctl}{\simrel_{\text{\tiny PCTL}}}
\newcommand{\simpctlnox}{\wsimrel_{\text{\tiny PCTL}_{{\setminus}X}}}
\newcommand{\simcsl}{\simrel_{\text{\tiny CSL}}}
\newcommand{\simcslnox}{\wsimrel_{\text{\tiny CSL}_{{\setminus}X}}}
\newcommand{\simpctlsafe}{\simrel_{\text{\tiny PCTL}}^{\text{\it \tiny safe}}}
\newcommand{\simpctllive}{\simrel_{\text{\tiny PCTL}}^{\text{\it \tiny live}}}
\newcommand{\simcslsafe}{\simrel_{\text{\tiny CSL}}^{\text{\it \tiny safe}}}
\newcommand{\simcsllive}{\simrel_{\text{\tiny CSL}}^{\text{\it \tiny live}}}
\newcommand{\simpctlnoxsafe}{\wsimrel_{\text{\tiny PCTL}_{{\setminus}X}}^{\text{\it \tiny safe}}}
\newcommand{\simpctlnoxlive}{\wsimrel_{\text{\tiny PCTL}_{{\setminus}X}}^{\text{\it \tiny live}}}
\newcommand{\simcslnoxsafe}{\wsimrel_{\text{\tiny CSL}_{{\setminus}X}}^{\text{\it \tiny safe}}}
\newcommand{\simcslnoxlive}{\wsimrel_{\text{\tiny CSL}_{{\setminus}X}}^{\text{\it \tiny live}}}

\newcommand{\Silent}{\mbox{\sl Silent}}

\newcommand{\bfR}{\RateMatrix}
\newcommand{\bfP}{\TransProb}
\newcommand{\Ratematrix}{\RateMatrix}
\newcommand{\simrel}{\prec}
\newcommand{\Distr}{\mbox{\sl Distr}}
\newcommand{\weightrel}{\sqsubseteq}


%\newcommand{\todo}[1]{\textbf{[\!\![\!\![~#1~]\!\!]\!\!]}}

\begin{document}
	
	\bibliographystyle{plain}

\title{Bisimulation minimisation mostly speeds up \\ probabilistic model checking}

%% alternative titles:  
%%      Bisimulation minimisation mostly in probabilistic verification
%%              

\author{Joost-Pieter Katoen$^{1,2}$, Tim Kemna$^2$, Ivan Zapreev$^{1,2}$ 
and David N.\ Jansen$^{2,1}$}
\institute{  
  {$^1$ Software Modeling and Verification Group, RWTH Aachen, Germany} \\
  {$^2$ Formal Methods and Tools, University of Twente, The Netherlands}
}

\maketitle
%
\begin{abstract}
This paper studies the effect of bisimulation minimisation in model checking of 
monolithic discrete-time and continuous-time Markov chains as well as variants 
thereof with rewards.  Our results show that---as for traditional model checking%
---enormous state space reductions (up to logarithmic savings) may be obtained.  
In contrast to traditional model checking, in many cases, the verification time of
the original Markov chain exceeds the quotienting time plus the verification time
of the quotient.  We consider probabilistic bisimulation as well as versions thereof
that are tailored to the property to be checked.
\end{abstract}

\section{Introduction}

Probabilistic model checking  enjoys a rapid increase of interest from different communities.  
Software tools such as PRISM \cite{HintonKNP_TACAS06} (with about 4,000 downloads),  MRMC \cite{KatoenKZ_QEST05}, and 
LiQuor \cite{BaierCG_SIGMETRICS05} 
support the verification of Markov chains or variants thereof that exhibit 
nondeterminism.
They have been applied to case studies from areas such as randomised distributed algorithms, 
planning and AI, security, communication protocols, biological process modeling, and quantum
computing.
Probabilistic model checking engines have been integrated in existing tool chains for widely
used formalisms such as stochastic Petri nets \cite{DAprileDS_DS04}, Statemate \cite{BodeHHJPPWB_QEST06}, and 
the stochastic process algebra PEPA \cite{Hillston_96}, and are used for a probabilistic extension of 
Promela \cite{BaierCG_SIGMETRICS05}. 

The typical kind of properties that can be checked is time-bounded reachability properties---``Does
the probability to reach a certain set of goal states (by avoiding bad states) within a maximal
time span exceed $\frac{1}{2}$?''---and long-run averages---``In equilibrium, does the likelihood to
leak confidential information remain below 10$^{{-}4}$?''
Extensions for cost-based models allow for checking more involved properties that refer to e.\,g.,
the expected cumulated cost or the instantaneous cost rate of computations. 
Intricate combinations of numerical or simulation techniques for Markov chains, optimisation 
algorithms, and traditional LTL or CTL model-checking algorithms result in simple, yet very
efficient verification procedures.
Verifying time-bounded reachability properties on models of tens of millions of states usually
is a matter of seconds.

Like in the traditional setting, probabilistic model checking suffers from state space 
explosion: the number of states grows exponentially in the number of system
components and cardinality of data domains.
To combat this problem, various techniques have been proposed in the literature.
Variants of binary decision diagrams (multi-terminal BDDs) have been
(and still are)
successfully
applied in PRISM \cite{HintonKNP_TACAS06} to a range of probabilistic models,  abstraction-refinement has been 
applied to reachability problems in MDPs~\cite{DArgenioJJL_PAPMPROBMIV01}, partial-order reduction techniques using 
Peled's ample-set method have been generalised to MDPs \cite{GroesserB_FMCOB05}, abstract interpretation 
has been applied to MDPs~\cite{Monniaux_SCP05}, and various bisimulation equivalences and simulation 
pre-orders allow model aggregation prior to model checking, e.\,g., \cite{BaierKHW_IC05,Sproston_TSE06}.
Recently proposed techniques include abstractions of probabilities by intervals 
combined with three-valued logics for DTMCs \cite{FecherLW_SPIN06,Huth_VSS04,Huth_TCS05}, stochastic ordering
techniques for CSL model checking \cite{MamounPY_EQST06}, abstraction of MDPs by two-player stochastic 
games \cite{KwiatkowskaNP_QEST06}, and symmetry reduction \cite{KwiatkowskaNP_CAV06}.

The purpose of this paper is to empirically investigate the effect of strong bisimulation minimisation 
in probabilistic model checking.
We hereby focus on fully probabilistic models such as discrete-time and continuous-time Markov 
chains (DTMCs and CTMCs, for short), and variants thereof with costs.
The advantages of probabilistic bisimulation \cite{LarsenS_IC91} in this setting are manifold.
It preserves the validity of PCTL \cite{HanssonJ_FAC94} and CSL \cite{AzizSSB_ACMTCL00,BaierHHK_TSE03} formulas, variants of CTL for the discrete- and continuous-time probabilistic setting, respectively.
It implies ordinary lumpability of Markov chains \cite{Buchholz_JAP94}, an aggregation technique for Markov chains that is applied in performance and dependability evaluation since the 1960s.
Quotient Markov chains can be obtained in a fully automated way.
The time complexity of quotienting is logarithmic in the number of states, and linear in the number 
of transitions---as for  traditional bisimulation minimisation---when using splay trees (a specific kind 
of balanced tree) for storing partitions \cite{DerisaviHS_IPL03}.
Besides, probabilistic bisimulation can be used for obtaining (coarser) abstractions that are
tailored to the properties of interest (as we will see), and enjoys the congruence property for
parallel composition allowing compositional minimisation.
We consider explicit model checking as the non-trivial interplay between bisimulation and 
MTBDDs would unnecessarily complicate our study; such symbolic representations mostly grow
under bisimulation minimisation~\cite{HermannsKNPS_JLAP03}. 

Thanks to extensive studies by Fisler and Vardi \cite{FislerV_FMCAD98,FislerV_CHARME99,FislerV_FMSD02}, it is known that 
bisimulation minimisation for LTL model checking and invariant verification leads to drastic state 
space reductions (up to logarithmic savings) but at a time penalty: the time to minimise and 
model check the resulting quotient Kripke structure significantly exceeds the time to verify 
the original model.
This paper considers these issues in probabilistic (i.\,e., PCTL and CSL) model checking.  
To that end, bisimulation minimisation algorithms have been realised in the prototypical 
explicit-state probabilistic model checker MRMC, several case studies have been considered that are
widely studied in the literature (and can be considered as benchmark problems), and
have been subjected to various experiments.
This paper presents our results.
As expected, our results show that enormous state space reductions (up to logarithmic savings) 
may be obtained.  
In contrast to the results by Fisler and Vardi \cite{FislerV_FMCAD98,FislerV_CHARME99,FislerV_FMSD02}, the verification time of 
the original Markov chain mostly \emph{exceeds} the quotienting time plus the verification time of 
the quotient.
	This effect is stronger for probabilistic bisimulation
	that is tailored to the property to be checked
	and for model checking Markov chains with costs (i.\,e.\@, rewards).
	This is due to the fact
	that probabilistic model checking is more time-consuming
	than traditional model checking,
	while minimization w.\,r.\,t.\@ probabilistic bisimulation
	is only slightly slower than for traditional bisimulation.

The paper is organised as follows.
Section~\ref{sec:basics} introduces the considered probabilistic models.
Section~\ref{sec:bisim} considers probabilistic bisimulation and the algorithms used.
Section~\ref{sec:exper} presents the considered case studies, the obtained results, and 
analyses these results.
Section~\ref{sec:concl} concludes the paper. 

\section{Preliminaries}
\label{sec:basics}

\paragraph{DTMCs.}

        Let $\AP$ be a fixed, finite set of \emph{atomic propositions.}
        A (labelled) DTMC $\D$ is a tuple $(S,\TransProb,L)$
        where $S$ is a finite set of \emph{states,}
        $\TransProb: S \times S \to [0,1]$ is a \emph{probability matrix}
        such that $\sum_{s' \in S} \TransProb(s,s') = 1$ for all $s \in S$,
        and $L : S \to 2^\mathit{AP}$ is a \emph{labelling} function
        which assigns to each state $s \in S$
        the set $L(s)$ of atomic propositions that hold in $s$.
        A path through a DTMC is a sequence\footnote{In
                this paper, we do not dwell upon the distinction between finite and infinite paths.}
        of states $\sigma = s_0 \, s_1 \, s_2 \ldots$
        with $\TransProb(s_i,s_{i{+}1}) > 0$ for all $i$.
        Let $\Path^\D$ denote the set of all paths in DTMC $\D$.
        $\sigma[i]$ denotes the $(i{+}1)$th state of $\sigma$,
        i.\,e.\@, $\sigma[i] = s_i$.
%       Let $\mathrm{Pr}_s$ denote the unique probability measure on sets of paths
%       that start in state $s$ \cite{...}.

\paragraph{The logic PCTL.}

        Let $a \in \AP$,
        probability $p \in [0,1]$,
        $k \in \mathbb{N}$ (or $k = \infty$)
        and $\bowtie$ be either $\leq$ or $\geq$.
        The syntax of Probabilistic CTL (PCTL) \cite{HanssonJ_FAC94} is defined by:
	\vspace{-1pt}
        \[ \Phi ::= \true \ \ \Big|  \ \ a \ \ \Big| 
              \ \ \Phi \, \wedge \, \Phi  \ \ \Big| 
              \ \ \neg \, \Phi \ \ \Big| 
              \ \ \mathcal{P}_{\bowtie \, p}(\Phi \rU{\leq k} \Phi) .
        \]
	\vspace{-1pt}
        A state $s$ satisfies $\mathcal{P}_{\bowtie \, p}(\Phi \rU{\leq k} \Psi)$
        if $\set{ \sigma \in \Path^\D(s) \mid \sigma \models \Phi \rU{\leq k} \Psi}$ 
        has a probability that satisfies $\bowtie \, p$.
        A path $\sigma$ satisfies $\Phi \rU{\leq k} \Psi$
        if within $k$ steps a $\Psi$-state is reached,
        and all preceding states satisfy $\Phi$.
        That is, if $\sigma[j] \models \Psi$ for some $j \leq k$,
        and $\sigma[i] \models \Phi$ for all $i < j$.
	We define the abbreviation $\Diamond^{\leq k} \Phi := \true \rU{\leq k} \Phi$.
        The unbounded until formula that is standard in temporal logics
        is obtained by taking $k = \infty$,
        i.\,e.\@, $\Phi \U \Psi = \Phi \rU{\leq \infty} \Psi$.\footnote{For
                simplicity, we do not consider the next operator.}

        Given a set $F$ of PCTL formulas,
        we denote with PCTL$_F$ the smallest set of formulas
        that contains $F$ and is closed under the PCTL operators $\wedge$, $\neg$, and $\mathcal{U}$.

\paragraph{Verifying hop-constrained probabilistic reachability.}

        PCTL model checking~\cite{HanssonJ_FAC94} is carried out in the same way
        as verifying CTL
        by recursively computing the set $\Sat(\Phi) = \set{s \in S \mid s \models \Phi}$.
        The probability of $\set{ \sigma \mid \sigma \models \Phi \rU{\leq k} \Psi}$
%       (needed for $\mathcal{P}_{\bowtie \, p}(\Phi \rU{\leq k} \Psi)$)
        is the least solution of the following linear equation system.
        Let
        $S_1 = \set{s \mid s \models \Psi}$,
        $S_0 = \set{s \mid s \models \neg \Phi \wedge \neg \Psi}$,
        and $S_? = \set{s \mid s \models \Phi \wedge \neg \Psi} = S \setminus (S_1 \cup S_0)$.
        \begin{equation*}\label{buntil}
                \mathit{Prob}^\D(s, \Phi \rU{\leq k} \Psi) = \begin{cases}
                1       & \text{if } s \in S_1 \\
                \sum\limits_{s' \in S} \TransProb(s,s') \cdot \mathit{Prob}^\D(s',\Phi \rU{\leq {k{-}1}} \Psi)
                        & \text{if } s \in S_? \wedge k>0 \\
                0       & \text{otherwise} \end{cases}
        \end{equation*}
        One can simplify this system by replacing $S_0$ by
        $U_0 = S_0 \cup \set{ s \in S_? \mid \neg\exists \sigma \in \Path^\D(s): \sigma \models \Phi \U \Psi}$.
        If $k = \infty$, one may also replace $S_1$ by
        $U_1 = S_1 \cup \set{ s \in S_? \mid \forall \sigma \in \Path^\D(s): \sigma \models \Phi \U \Psi}$.
        The sets $U_0$ and $U_1$ can be found via a simple graph analysis (a depth-first search)
        in time $O(|S|{+}|\TransProb|)$.
%       This probability can be computed directly by solving the linear equation system
%       % by standard means such as Gaussian elimination
%       or can be approximated by an iteration\footnote{The
%               function $s \mapsto {\mathit Prob}^\D(s, \Phi \U^{\leq k} \Psi)$ is the least fixpoint
%               of a higher-order function in $(S \to [0,1]) \to (S \to [0,1])$
%               where the underlying partial order on $S \to [0,1]$ is defined pointwise.}.
%       \todo{I think that approximation is only necessary if $k=\infty$?}

	Alternatively, the probabilities can be calculated
        by making the states $s \not\in S_?$ absorbing as follows.
        For DTMC $\D = (S,\TransProb,L)$ and $A \subseteq S$,
        let $\D[A]$ be the DTMC $(S,\TransProb[A],L)$
        where the states in $A$ are made absorbing:
        If $s \in A$, then $\TransProb[A](s,s) = 1$
        and $\TransProb[A](s,s') = 0$ for $s' \neq s$.
        Otherwise, $\TransProb[A](s,s') = \TransProb(s,s')$.
%       We have $\D[A][B] = \D[A \cup B]$.
        Let $\pi^\D(s \overset{k}{\rightsquigarrow} s')$
        denote the probability of being in state $s'$ after exactly $k$ steps in DTMC $\D$
        when starting in $s$.
%       This is known as the transient probability of state $s'$ after $k$ steps
%       and can be obtained by $(\alpha_s{\cdot}\TransProb^k)(s')$
%       where $\alpha_s$ is a probability vector that is one in state $s$, and 0 otherwise.
        Then:
        \begin{equation*}\label{A}
                \mathit{Prob}^\D(s,\Phi \rU{\leq k} \Psi) =
                \sum_{s' \in S_1} \ \pi^{\D[S_0 \cup S_1]}(s \overset{k}{\rightsquigarrow} s') .
        \end{equation*}
%       Note that $\D[\neg \Phi \vee \Psi] = \D[\neg(\Phi \wedge \Psi)][\Psi]$,
%       i.\,e.\@, all $\neg(\Phi \vee \Psi)$-states and all $\Psi$-states in $\D$ are made absorbing.
%       That is, the only transitions available in these states
%       are self-loops with probability one.
%       The former is correct since $\Phi \rU{\leq k} \Psi$ is violated
%       as soon as some state is visited that neither satisfies $\Phi$ nor $\Psi$.
%       The latter is correct since, once a $\Psi$-state in $\D$ has been reached
%       (along a $\Phi$-path) in at most $k$ steps,
%       then $\Phi \rU{\leq k} \Psi$ holds,
%       regardless of which states will be visited later on.
        Calculating $\mathit{Prob}^\D(s,\Phi \rU{\leq k} \Psi)$
        thus amounts to computing $(\TransProb[S_0 \cup S_1])^k{\cdot}\vt{\iota_{S_1}}$,
        where $\vt{\iota_{S_1}}(s) = 1$ if $s \in S_1$, and 0 otherwise.
%       As iterative squaring is not attractive for stochastic matrices due to fill in~\cite{Stew94},
%       the product is typically computed in an iterative fashion:
%       $\TransProb{\cdot}( \ldots ( \TransProb{\cdot}\vt{\iota}_\Psi ))$.

\paragraph{CTMCs.}

        A (labelled) CTMC $\M$ is a tuple $(S,\TransProb,E,L)$
        where $(S,\TransProb,L)$ is a DTMC
        and $E: S \to \mathbb{R}_{\geq 0}$ provides the \emph{exit rate} for each state.
        The probability of taking a transition from $s$ within $t$ time units 
        equals $1-e^{{-}E(s) \cdot t}$.
        The probability of taking a transition from state $s$ to state $s'$
        within time $t$ is given by: $\TransProb(s,s') \cdot(1-e^{{-}E(s) \cdot t})$.

	A path through a CTMC is a sequence of states and sojourn times
	$\sigma = s_0 \, t_0 \, s_1 \, t_1 \ldots$
	with $\TransProb(s_i,s_{i+1})>0$ and $t_i \in \mathbb{R}_{\geq 0}$ for all $i$.
	Let $\Path^\M$ denote the set of all paths in CTMC $\M$.

\paragraph{Uniformisation.}
        In a \emph{uniform} CTMC, the exit rate of all states is the same.
        A non-uniform CTMC can be uniformized by adding self loops as follows:
        let $\M = (S,\TransProb,E,L)$ be a CTMC
        and choose $\tilde{E} \geq \max_{s \in S} E(s)$.
        Then, $\mathit{Unif}_{\tilde{E}}(\M) = (S,\TransProb',E',L)$
        where $E'(s) = \tilde{E}$ for all $s$,
        $\TransProb'(s,s') = E(s) \TransProb(s,s') / \tilde{E}$ if $s \not= s'$
        and $\TransProb'(s,s) = 1 - \sum_{s' \not= s} \TransProb'(s,s')$.
%       = [\tilde{E} - E(s)(1 - \TransProb(s,s))] / \tilde{E}$.
        The probability to be in a given state at a given time in the uniformized CTMC
	is the same as the one in the original CTMC.

\paragraph{The logic CSL.}

        Continuous stochastic logic (CSL, \cite{BaierHHK_TSE03}) is similar to PCTL.
        For $a$, $p$ and $\bowtie$ as before, time bounds $t_1 \in [0, \infty)$ and $t_2 \in [t_1, \infty]$,
        the syntax is:
        \[ \Phi ::= \true \ \ \Big|  \ \ a \ \ \Big| 
              \ \ \Phi \, \wedge \, \Phi  \ \ \Big| 
              \ \ \neg \, \Phi \ \ \Big| 
              \ \ \mathcal{P}_{\bowtie \, p}(\Phi \rU{[t_1, t_2]} \Phi)
                \ \ \Big|
                \ \ \mathcal{S}_{\bowtie \, p}(\Phi)
        \]
        A state $s$ satisfies $\mathcal{P}_{\bowtie \, p}(\Phi \rU{[t_1, t_2]} \Psi)$
        if the set of timed paths $\set{ \sigma \in \Path^\M(s) \mid \sigma \models \Phi \rU{[t_1, t_2]} \Psi }$
        has a probability $\bowtie \, p$.
        A timed path
        %$s_0 t_0 \, s_1 t_1 \, s_2 t_2 \ldots$
        satisfies $\Phi \rU{[t_1, t_2]} \Psi$
        if within time $t \in [t_1, t_2]$ a $\Psi$-state is reached,
        and all preceding states satisfy $\Phi$.
        We will mostly let $t_1=0$
        and denote this as $\Phi \rU{\leq t_2} \Psi$.
        A state $s$ satisfies the formula $\mathcal{S}_{\bowtie \, p}(\Phi)$
        if the steady-state probability
        to be in a $\Phi$-state (when starting in $s$)
        satisfies the constraint $\bowtie \, p$.

        CSL model checking \cite{AzizSSB_ACMTCL00,BaierHHK_TSE03} can be implemented as follows.
        The operator $\mathcal{S}$ can be solved by a (standard) calculation of the steady-state 
        probabilities together with a graph analysis.
        For the time-bounded until operator, note that,
        after uniformisation the probability to take $k$ steps within time $t$
        does not depend on the actual states visited.
        This probability is Poisson distributed,
        and the probability to satisfy the until formula within $k$ steps
        is calculated using the PCTL algorithm.
        The total probability is an infinite sum over all $k$, which can be approximated well.

\paragraph{Rewards.}
A discrete-time Markov reward model (DMRM) $\D_\rew$ is a tuple $(\D, \rew)$ where 
$\D$ is a DTMC and $\rew: S \to \mathbb{R}_{\geq 0}$ is a \emph{reward} assignment
function.
The quantity $\rew(s)$ indicates the reward that is earned on leaving state $s$.
Rewards could also be attached to edges in a DTMC, but this does not
increase expressivity.
A path through a DMRM is a path through its DTMC, i.\,e., sequence of states $\sigma = 
s_0 \, s_1 \, s_2 \ldots$ with $\TransProb(s_i,s_{i{+}1}) > 0$ for all $i$.

        Let $a$, $p$ and $k$ be as before,
        and $r \in \mathbb{R}_{\geq 0}$ be a nonnegative reward bound.
        The two main operators that extend PCTL to Probabilistic Reward CTL (PRCTL) \cite{AndovaHK_FORMATS03} 
        are $\mathcal{P}_{\bowtie \, p}(\Phi \mathrel{\U^{\leq k}_{\leq r}} \Psi)$ and
          $\mathcal{E}^{{=}k}_{\leq r}(\Phi)$.
        The until-operator is equipped with a bound
        on the maximum number ($k$) of allowed hops to reach the goal states,
        and a bound on the maximum allowed cumulated reward ($r$)
        before reaching these states.
Formula $\mathcal{E}^{{=}k}_{\leq r}(\Phi)$ asserts that the expected cumulated 
reward in $\Phi$-states until the $k$-th transition is at most $r$.
Thus, in order to check the validity of this formula for a given path, all visits to
$\Phi$-state are considered in the first $k$ steps and the total reward that is 
obtained in these states;
        the rewards earned in other states
        or earned in $\Phi$-states after the first $k$ steps
        are not relevant.
Whenever the expected value of this quantity over all paths that start in state 
$s$ is at most $r$, state $s \models \mathcal{E}^{{=}k}_{\leq r}(\Phi)$.

A continuous-time Markov reward model (CMRM) $\M_\rew$ is a tuple $(\M, \rew)$ 
where $\M$ is a CTMC and $\rew: S \to \mathbb{R}_{\geq 0}$ is a reward assignment
function (as before).
The quantity $\rew(s)$ indicates that if $t$ time units are spent in state $s$, a reward 
$\rew(s)\cdot t$ is acquired.
A path through a CMRM is a path through its underlying CTMC.
Let $\sigma = s_0 \, t_0 \, s_1 \, t_1 \ldots$ be a path. 
For $t = \sum_{j{=}0}^{k{-}1} t_j + t'$ with $t' \leq t_k$ we define 
$\rew(\sigma,t) = \sum_{j{=}0}^{k{-}1} t_j \cdot \rew(s_j) + t' \cdot \rew(s_k)$, 
the cumulative reward along $\sigma$ up to time $t$.

	CSRL \cite{BaierHHK_ICALP00} is a logic that extends CSL with one operator
	$\mathcal{P}_{\bowtie\, p}(\Phi \mathrel{\U^{\leq t}_{\leq r}} \Psi)$
	to express time- and reward-bounded properties.
	Checking this property of a CMRM is difficult.
        One can either approximate the CMRM by a discretisation of the rewards
        or compute for each (untimed) path the probability to meet the bound
        and sum them up.
        Reward-bounded until properties of a CMRM can be checked via a transformation
        of rewards into exit rates and checking a corresponding time-bounded until property
        \cite{BaierHHK_ICALP00}.

\section{Bisimulation}
\label{sec:bisim}

\paragraph{Bisimulation.}
        Let $\D = (S,\TransProb,L)$ be a DTMC and $R$ an equivalence relation on $S$.
        The quotient of $S$ under $R$ is denoted $S/R$.
        $R$ is a \emph{strong bisimulation} on $\D$ if for $s_1 \mathrel{R} s_2$: 
        \[
                L(s_1) = L(s_2) \quad \mbox{ and } \quad \TransProb(s_1,C) = 
                \TransProb(s_2,C) \mbox{ for all } C \mbox{ in } S/R.
        \]
        $s_1$ and $s_2$ in $\D$ are strongly bisimilar, denoted $s_1 
        \sim_d s_2$, if there exists a strong bisimulation $R$ on $\D$ with 
        $s_{1} \mathrel{R} s_{2}$.
        Strong bisimulation~\cite{Buchholz_JAP94,Hillston_96} for CTMCs,
        that implies ordinary lumpability,
        is a mild variant of the notion for the discrete-time probabilistic setting:
        in addition to the above, it is also required
        that the exit rates of bisimilar states are equal:
        $E(s_1) = E(s_2)$.

\paragraph{Measure-driven bisimulation.}
        Requiring states to be equally labelled with all atomic propositions is rather
        strong if one is interested in checking formulas that just refer to a (small)
        subset of propositions, or more generally, sub-formulas.
        The following notion weakens the labelling requirement in strong bisimulation 
        by requiring equal labellling for a set of PCTL formulas $F$.
        Let $\D = (S,\TransProb,L)$ be a DTMC
        and $R$ an equivalence relation on $S$.
        $R$ is a \emph{$F$-bisimulation} on $\D$ if for $s_1 \mathrel{R} s_2$:
        \[
                s_1 \models \Phi \Longleftrightarrow s_2 \models \Phi
                \text{ for all } \Phi \in F
                \quad \text{ and } \quad
                \TransProb(s_1, C) = \TransProb(s_2, C)
                \text{ for all } C \in S/R.
        \]
        States $s_1$ and $s_2$ are $F$-bisimilar, denoted $s_1 \mathrel{\sim_F} s_2$,
        if there exists an $F$-bisimulation $R$ on $\D$ with $s_1 \mathrel{R} s_2$.
        $F$-bisimulation on CTMCs (for a set of CSL formulas $F$) is defined analogously
        \cite{BaierHHK_ICALP00}.
        Note that strong bisimilarity is $F$-bisimilarity for $F = \AP$.

\paragraph{Preservation results.}
        Aziz \emph{et al.} \cite{AzizSSBSV_CAV95} have shown that strong bisimulation is sound and complete 
        with respect to PCTL (and even PCTL$^*$):
        \begin{proposition}
                Let $\D$ be a DTMC, $R$ a bisimulation
                and $s$ an arbitrary state of $\D$.
                Then, for all PCTL formulas $\Phi$,
                $s \models_\D \Phi \Longleftrightarrow [s]_R \models_{\D/R} \Phi$.
        \end{proposition}
        This result can be generalised to $F$-bisimulation in the following way:
        \begin{proposition} \label{prop.f.bisim}
                Let $\D$ be a DTMC, $R$ an $F$-bisimulation
                and $s$ an arbitrary state of $\D$.
                Then, for all PCTL$_F$ formulas $\Phi$,
                $s \models_\D \Phi \Longleftrightarrow [s]_R \models_{\D/R} \Phi$.
        \end{proposition}
        Similar results hold for CSL and bisimulation on CTMCs \cite{BaierHHK_TSE03},
        for PRCTL on DMRM, and for CSRL on CMRM.
        
%       \todo{is it somewhere explained what is PCTL$_F$?}
%       Yes, just below the definition of PCTL.

\paragraph{Bisimulation minimisation.}
        The preservation results suggest that one can verify properties of a Markov chain
        on a bisimulation quotient.
        The next issue to consider is how to obtain the quotient.
        An often used algorithm (called \emph{partition refinement}) is based on \emph{splitting:}
        Let $\Pi$ be a partition of $S$.
%       (The initial partition is constructed from the state labelling.)
        A splitter for some block $B \in \Pi$ is a block $Sp \in \Pi$
        such that the probability to enter $Sp$ is not the same for each state in $B$.
        In this case, the algorithm splits $B$ into subblocks
        such that each subblock consists of states $s$ with identical $\TransProb(s,Sp)$.
        This step is repeated until a fixpoint is reached.
        The final partition is the coarsest bisimulation that respects the initial partition.
        The worst-case time complexity of this algorithm is $O(|\TransProb| \log |S|)$ provided
        that splay trees are used to store blocks \cite{DerisaviHS_IPL03}.
        These data structures are adopted in our implementation.\footnote{
                In practice, an implementation using red-black trees is often slightly faster,
                although this raises the theoretical complexity to 
                $O(|\TransProb| \log^2 |S|)$, cf.\@ \cite[Section~3.4]{Derisavi_PhD05}.}

\paragraph{Initial partition.}
        The choice of initial partition in the partition refinement algorithm
        determines what kind of bisimulation the result is.
        If we group states labelled with the same atomic propositions together,
%       (this can be done in time $O(|S|{\cdot}|\AP|)$),
        the result is the strong bisimulation quotient $S/\mathord{\sim}_d$.
        If we choose the initial partition according to the satisfaction of formulas in $F$,
%       (based on earlier model checking steps),
        the resulting partition is the $F$-bisimulation quotient $S/\mathord{\sim}_F$.
        To get the smallest bisimulation quotient,
        it is important to start with a coarse initial partition.
        Instead of only calculating the strong bisimulation quotient,
        we will also use measure-driven bisimulation for a suitable set $F$.

        A naive approach for formula $\mathcal{P}_{\bowtie \, p}(\Phi \U \Psi)$
        is to choose $F = \set{ \Psi, \Phi \wedge \neg \Psi }$.
        In fact, $\mathcal{P}_{\bowtie \, p}(\Phi \U \Psi)$ is not in PCTL$_F$, but the equivalent
        formula $\mathcal{P}_{\bowtie \, p}(\Phi \wedge \neg \Psi \U \Psi)$ is.
        This yields an initial partition consisting of the sets $S_1 = \Sat(\Psi)$, $S_? = \Sat(\Phi
        \wedge \neg \Psi)$ and $S_0 = S \setminus (S_1 \cup S_?)$
	(cf.\@ Section~\ref{sec:basics}).
	Note that selecting $F = \set{ \Psi, \Phi }$
	would lead to a less efficient initial partition
	with four blocks instead of three.
%       The sets $S_1$ and $S_0$ may be collapsed into a single state each.
        We improve this initial partition by replacing $S_0$
        by $U_0 = \Sat(\mathcal{P}_{\leq 0}(\Phi \U \Psi))$
        and $S_1$ by $U_1$,
        which is essentially\footnote{
                Up to states $s$ where the set
                $\set{ \sigma \in \Path^\D(s) \mid \sigma \not\models \Phi \U \Psi }$
                is only almost empty.
%               Then, $s \not\in U_1$.
%
%               Note that $U_0$ is actually, not only almost $= \Sat{\mathcal{P}_{\leq 0}(\Phi \U \Psi)$
%               because if $\sigma \models \Phi \U \Psi$,
%               then a finite initial fragment of $\sigma$ also satisfies $\Phi \U \Psi$,
%               and the set of paths with this initial fragment has measure $> 0$.
                }
        $\Sat(\mathcal{P}_{\geq 1}(\Phi \U \Psi))$.
%	This corresponds to $F = \set{ \mathcal{P}_{\leq 0}(\Phi \U \Psi), \mathcal{P}_{\geq 1}(\Phi \U \Psi)}$,
%	and we check the scary-looking but still equivalent formula
%	$\mathcal{P}_{\bowtie \, p}(\neg \mathcal{P}_{\leq 0}(\Phi \U \Psi) \U \mathcal{P}_{\geq 1}(\Phi \U \Psi))$.
	(Defining $U_0$ and $U_1$ as satisfaction sets of some formula has the advantage
	that we can still use Proposition~\ref{prop.f.bisim}.)
        The sets of states $U_0$ and $U_1$
        can be collapsed into single states $u_0$ and $u_1$, respectively.
        This results in the initial partition $\set{ \{u_0\}, \{u_1\}, S \setminus (U_0 \cup U_1)}$.

        For bounded until, one can still use $U_0$, but not $U_1$,
        since the fact that (almost) all paths satisfy $\Phi \U \Psi$
        does not imply that these paths reach a $\Psi$-state within the step or time bound.
        Therefore, for this operator the initial partition is $\set{ \{u_0\}, \{s_1\}, S \setminus (U_0 \cup S_1)}$
        with $u_0$ as before and $s_1$ the collapsed state for $S_1$.\footnote{
                For the sake of brevity,
                we omit the details for the optimal initial partition for time-bounded until-formulas
                of the form $\rU{[t_1,t_2]}$ with $0 < t_1$.
                }
        Thus, for bounded until the measure-driven initial partition is finer than for unbounded until.
        In the experiments reported in the next section, the effect of the granularity of the initial
        partition will become clear.

\section{Experiments}
\label{sec:exper}

To study the effect of bisimulation in model checking,
we realised the minimisation algorithms in MRMC
and applied them to a variety of case studies, most of which  
can be obtained from the PRISM webpage.\footnote{see {\tt http://www.cs.bham.ac.uk/dxp/prism/index.php}.}
We used PRISM to specify the models and generate the Markov chains.
Subsequently, the time and memory requirements have been considered for verifying
the chains (by MRMC), and for minimising plus verifying the lumped chain (both by 
MRMC). 
All experiments were conducted on a 2.66~GHz Pentium~4 processor with 1~GB RAM running 
Linux.
All reported times are in milliseconds and are obtained by taking the average of running 
the experiment 10 times.

\subsection{Discrete time}

\paragraph{Crowds protocol \cite{ReiterR_ACMTISS98}.}
This protocol uses random routing within a group of nodes (a crowd) to establish a 
connection path between a sender and a receiver.  
Routing paths are reconstructed once the crowd changes; the number of such new route
establishments is $R$, and is an important parameter that
influences the state space.
Random routing serves to hide the secret identity of a sender.
The table below summarises the results for checking $\Prob{\leq}{p}{\Diamond \mbox{\sl 
observe}}$ where {\sl observe} characterises a situation in which the sender's id is
detected.
The parameter $N$ in the first column is the number of honest crowd members;
our models include $N/5$ dishonest members.
The second column shows parameter $R$.
The next three columns indicate the size of the state space of the DTMC (i.\,e., $|S|$),
the number of transitions (i.\,e., the number of non-zero entries in $\TransProb$), and the 
verification time.
The next three columns indicate the number of states in the quotient DTMC, the time needed
for obtaining this quotient, and the time to check the validity of the same formula on
the quotient.
The last two columns indicate the reduction factor for the number of states and total time.
Note that we obtain large state space reductions.  
Interestingly, in terms of time consumption, quotienting obtains a reduction in time of
about a factor 4 to 7.

{\scriptsize
\begin{center}
\begin{tabular}{|c|c|r|r|r|r|r|r|r|r|}
\hline 
%% \hline
%% \multicolumn{10}{|c|}{\emph{Reiter and Rubin's Crowds protocol}} \\ 
%% \hline \hline
\multicolumn{2}{|c}{} & \multicolumn{3}{c|}{original DTMC} & \multicolumn{3}{c|}{lumped DTMC} 
  & \multicolumn{2}{c|}{red.\ factor} \\[1ex] \hline 
     $N$    &$R$& \multicolumn{1}{c|}{states} &transitions& ver.\ time    &blocks& lump time     &ver.\ time& \multicolumn{1}{c|}{states} & time \\ \hline
\phantom{0}5& 3 &    1198 &    2038 &     3.2          &  53 &    0.6          & 0.3 &   22.6          & 3.7 \\
\phantom{0}5& 4 &    3515 &    6035 &    11\phantom{.0}&  97 &    2.0          & 0.5 &   36.2          & 4.4 \\
\phantom{0}5& 5 &    8653 &   14953 &    48\phantom{.0}& 153 &    6.0          & 0.9 &   56.6          & 6.9 \\
\phantom{0}5& 6 &   18817 &   32677 &   139\phantom{.0}& 209 &   14\phantom{.0}& 1.4 &   90.0          & 9.0 \\ \hline 
% ----------------------------------------------------------------
         10 & 3 &    6563 &   15143 &    24\phantom{.0}&  53 &    4.6          & 0.2 &  124\phantom{.0}& 4.9 \\
         10 & 4 &   30070 &   70110 &   190\phantom{.0}&  97 &   29\phantom{.0}& 0.5 &  310\phantom{.0}& 6.4 \\
         10 & 5 &  111294 &  261444 &   780\phantom{.0}& 153 &  127\phantom{.0}& 0.9 &  727\phantom{.0}& 6.1 \\
         10 & 6 &  352535 &  833015 &  2640\phantom{.0}& 221 &  400\phantom{.0}& 1.4 & 1595\phantom{.0}& 6.6 \\ \hline 
% --------------------------------------------------------------------------------------
         15 & 3 &   19228 &   55948 &   102\phantom{.0}&  53 &   23\phantom{.0}& 0.2 &   363\phantom{.0}& 4.4 \\
         15 & 4 &  119800 &  352260 &   790\phantom{.0}&  97 &  190\phantom{.0}& 0.5 &  1235\phantom{.0}& 4.1 \\
         15 & 5 &  592060 & 1754860 &  4670\phantom{.0}& 153 & 1020\phantom{.0}& 0.9 &  3870\phantom{.0}& 4.6 \\
         15 & 6 & 2464168 & 7347928 & 20600\phantom{.0}& 221 & 4180\phantom{.0}& 1.5 & 11150\phantom{.0}& 4.9 \\ \hline 
\end{tabular}
\end{center}
}

\paragraph{Leader election \cite{ItaiR_IC90}.}
In this protocol, $N$ nodes that are arranged in an unidirectional ring select
an identity randomly according to a uniform distribution on $\set{1, \ldots, K}$.
By means of synchronous message passing, processes send their identity around
the ring.  The protocol terminates once a node has selected a unique id (the 
node with the highest unique id becomes the leader); if no such node exists, the 
protocol restarts.
The property of interest is the probability to elect a leader
within a certain number of rounds: $\Prob{\leq}{q}{\Diamond^{\leq (N{+}1){\cdot}3} \, 
\mbox{\sl leader elected}}$.
The obtained results are summarised in the table below.
For a fixed $N$, the number of blocks is constant.
This is due to the fact that the initial state is the only probabilistic state 
and that almost all states that are equidistant w.\,r.\,t.\ this initial state are bisimilar.
For $N = 4$, no gain in computation time is obtained due to the relatively 
low number of iterations needed in the original DTMC.
When $N$ increases, bisimulation minimisation also pays off timewise; in this
case a small reduction of the time is obtained (more iterations are needed
due to the bound in the until-formula that depends on $N$). 

{\scriptsize
\begin{center}
\begin{tabular}{|c|r|r|r|r|r|r|r|r|r|}
\hline 
%% \hline
%% \multicolumn{10}{|c|}{\emph{Itai and Rodeh's synchronized leader election protocol}} \\ 
%% \hline \hline
\multicolumn{2}{|c}{} & \multicolumn{3}{c|}{original DTMC} & \multicolumn{3}{c|}{lumped DTMC} 
  & \multicolumn{2}{c|}{red.\ factor} \\[1ex] \hline 
$N$&$K$& \multicolumn{1}{c|}{states}&transitions& ver.\ time   &blocks& lump time   &ver.\ time& \multicolumn{1}{c|}{states} &time \\ \hline
4 &  2 &     55 &     70 &   0.02          & 10 &   0.05          & 0.01 &     5.5          & 0.4 \\
4 &  4 &    782 &   1037 &   0.4\phantom{0}& 10 &   0.5\phantom{0}& 0.01 &    78.2          & 0.8 \\
4 &  8 &  12302 &  16397 &   7.0\phantom{0}& 10 &   9.0\phantom{0}& 0.01 &  1230\phantom{.0}& 0.8 \\
4 & 16 & 196622 & 262157 & 165.0\phantom{0}& 10 & 175\phantom{.00}& 0.01 & 19662\phantom{.0}& 0.9 \\ \hline 
5 &  2 &    162 &    193 &   0.1\phantom{0}& 12 &   0.1\phantom{0}& 0.02 &    13.5          & 0.9 \\
5 &  4 &   5122 &   6145 &   2.8\phantom{0}& 12 &   2.9\phantom{0}& 0.02 &   427\phantom{.0}& 0.9 \\
5 &  6 &  38882 &  46657 &  28\phantom{.00}& 12 &  26\phantom{.00}& 0.02 &  3240\phantom{.0}& 1.1 \\
5 &  8 & 163842 & 196609 & 140\phantom{.00}& 12 & 115\phantom{.00}& 0.02 & 13653\phantom{.0}& 1.2 \\ \hline 
\end{tabular}
\end{center}
}

\paragraph{Cyclic polling server \cite{IbeT_SAC90}.}
This standard example in performance analysis considers a set of stations that
are allowed to process a job once they possess the token.
The single token circulates among the stations.
The times for passing a token to a station and for serving a job are all distributed 
exponentially.
We consider the DTMC that is obtained after uniformisation, and check the
formula: $\Prob{\bowtie}{p}{\bigwedge_{j \neq 1}^N \neg \mbox{\sl serve}_j \U \mbox{\sl serve}_1}$, 
i.\,e.\@ with probability $\bowtie p$ station 1 will be served before any other 
station, as well as a time-bounded version thereof.\footnote{For the sake of
comparison, the unbounded until-formula is checked on the uniformised
and not on the embedded DTMC.}
Ordinary (strong) bisimulation yields no state-space reduction.
The results for measure-driven bisimulation minimisation are summarised
below.
In checking the bounded until formula, we used the naive initial partition 
$\set{ \{s_0\}, \{s_1\}, S_? }$. 
The improved initial partition with $\{u_0\}$ would have led to almost the same 
number of blocks as the unbounded until, e.\,g. 46 instead of 151 blocks for $N=15$.
For both formulas, large reductions in state space size as well as computation 
time are obtained; the effect of $\{u_0\}$ on the number of blocks is also considerable.

{\scriptsize
\begin{center}
\begin{tabular}{|c|r|r|r|r|r|r|r|r|r|r|r|r|}
\hline 
%% \hline
%% \multicolumn{11}{|c|}{\emph{Trivedi and Ibe's cyclic polling protocol}}\ 
%% \hline \hline
  \multicolumn{5}{|c|}{} & \multicolumn{4}{c|}{time-bounded until}
  & \multicolumn{4}{c|}{unbounded until} \\ \hline
  \multicolumn{5}{|c|}{original DTMC} & \multicolumn{2}{c|}{lumped DTMC}  
  & \multicolumn{2}{c|}{red.\ factor}  & \multicolumn{2}{c|}{lumped DTMC}  
  & \multicolumn{2}{c|}{red.\ factor}\\[1ex] \hline 
$N$& \multicolumn{1}{c|}{states} &transitions&time $\U^{\leq t}$& \multicolumn{1}{c|}{time $\U$} & blocks & \multicolumn{1}{c|}{time} & states & time 
  & blocks & time & states & time \\ \hline
 4&    96&    368&    1.4          &   2.1           &  19&   0.4          &   5.1          & 3.5&12&  0.9          &    8& 2.3 \\
 6&   576&   2784&   10\phantom{.0}&   11\phantom{.0}&  34&   1.2          &  16.9          & 8.3&18&  1.4          &   32& 7.9 \\
 8&  3072&  17920&   62\phantom{.0}&   52\phantom{.0}&  53&   4.0          &  58\phantom{.0}&15.5&24&  2.9          &  128&17.9  \\
12& 73728& 577536& 3050\phantom{.0}&  3460\phantom{.0} &103 & 120\phantom{.0}& 716\phantom{.0}&25.4&36& 55\phantom{.0}& 2048&62.9  \\
15&737280&6881280&39000\phantom{.0}&32100\phantom{.0}& 151&1590\phantom{.0}&4883\phantom{.0}&24.5&45&580\phantom{.0}&16384&55.3 \\ \hline
\end{tabular}
\end{center}
}

\paragraph{Randomised mutual exclusion \cite{PnueliZ_DC86}.}
%
In this mutual exclusion algorithm, $N$ processes make random choices based on coin tosses 
to ensure that they can all enter their critical sections eventually, although not simultaneously.
The following table summarizes our results for verifying the property that process~1
is the first to enter the critical section, i.\,e., the PCTL formula $\Prob{\leq}{q}{\bigwedge_{j \neq 1}^N
\neg \mbox{\sl enter}_j \U \mbox{\sl  enter}_1}$.

{\scriptsize
\begin{center}
\begin{tabular}{|c|r|r|r|r|r|r|r|r|r|r|r|r|r|}
\hline 
%% \hline
%% \multicolumn{12}{|c|}{\emph{Pnueli and Zuck's randomized mutual exclusion protocol}} \\ \hline
   & \multicolumn{3}{c|}{}          & \multicolumn{5}{c|}{strong bisimulation}      & \multicolumn{4}{c|}{$F$-bisimulation} \\
\hline %----------------------------------------------------------------------------------------------------------------
   & \multicolumn{3}{c|}{original DTMC}
                                    & \multicolumn{3}{c|}{lumped DTMC}   & \multicolumn{2}{c|}{red.\ factor}
                                                                                    & \multicolumn{2}{c|}{lumped DTMC}  
                                                                                                            & \multicolumn{2}{c|}{red.\ factor}\\[1ex]
\hline %----------------------------------------------------------------------------------------------------------------
$N$& \multicolumn{1}{c|}{states} &transitions& ver.\ time  &blocks& lump time     & ver.\ time  &states&time&blocks& \multicolumn{1}{c|}{time} &states&time\\
\hline %----------------------------------------------------------------------------------------------------------------
3&   2368&    8272&    3.0          & 1123&   8.0          &  1.6          & 2.1&0.3&   233 &    2.9          & 10.2          &1.0 \\
4&  27600&  123883&   47.0          & 5224& 192\phantom{.0}& 19\phantom{.0}& 5.3&0.4&   785 &   29\phantom{.0}& 35.2          &1.6 \\
5& 308800& 1680086&  837\phantom{.0}&18501&2880\phantom{.0}&120\phantom{.0}&16.7&0.3&  2159 &  507\phantom{.0}&143\phantom{.0}&1.7 \\
6&3377344&21514489& 9589\phantom{.0}&  -- &  $> 10^7$ &  --                & -- & --&  5166 & 7106\phantom{.0} & 653\phantom{.0}& 1.4 \\ \hline
\end{tabular}
\end{center}
}
%
\noindent Due to the relatively high number of transitions, quotienting the DTMC according to $\AP$-bisimilarity
is computationally expensive, and takes significantly more time than verifying the original DTMC.
However, measure-driven bisimilarity yields a quotient that is roughly an order of magnitude smaller
than the quotient under $\AP$-bisimilarity.
Due to the coarser initial partition, this quotient is constructed rather fast.
In this case, verifying the original model is more time consuming.

\subsection{Continuous time}

%This subsection reports on the results obtained for the verification of CTMCs.

\paragraph{Workstation cluster \cite{HaverkortHK_SRDS00}.}
This case study considers a system consisting of two clusters of workstations connected
via a backbone.  Each cluster consists of $N$ workstations, connected in a star topology
with a central switch that provides the interface to the backbone.  Each component can 
break down according to a failure distribution.  A single repair unit is available to repair
the failed components.  The number of correctly functioning workstations determines the
level of quality of service (QoS).  
The following two tables summarise the results for checking the probability that:
\begin{itemize}
\item In the long run, premium QoS will be delivered in at least 70\% of the cases;
\item QoS drops below minimum QoS within 40 time-units is at most 0.1;
\item QoS goes from minimum to premium between 20 and 40 time units.
\end{itemize}
The last property involves a sequence of two transient analyses on different CTMCs.
The results for the long-run property:
%
{\scriptsize
\begin{center}
\begin{tabular}{|r|r|r|r|r|r|r|r|r|}
\hline 
%% \hline
%% \multicolumn{10}{|c|}{\emph{Long-run property for workstation  cluster}} \\ 
%% \hline \hline
  & \multicolumn{3}{c|}{original CTMC} & \multicolumn{3}{c|}{lumped CTMC} 
  & \multicolumn{2}{c|}{red.\ factor} \\[1ex] \hline 
\multicolumn{1}{|c|}{$N$} & \multicolumn{1}{c|}{states} &transitions& ver.\ time      & \multicolumn{1}{c|}{blocks} &lump time&ver.\ time&states&time \\ \hline
  8 &    2772 &    12832 &     3.6          &    1413 &    12 &    130   &  2  & 0.03 \\
 16 &   10132 &    48160 &    21\phantom{.0}&    5117 &    64 &    770   &  2  & 0.03 \\
 32 &   38676 &   186400 &   114\phantom{.0}&   19437 &   290 &    215   &  2  & 0.2\phantom{0} \\
 64 &  151060 &   733216 &   730\phantom{.0}&   75725 &  1360 &   1670   &  2  & 0.2\phantom{0} \\
128 &  597012 &  2908192 &  6500\phantom{.0}&  298893 &  5900 &  14900   &  2  & 0.2\phantom{0} \\
256 & 2373652 & 11583520 &103000\phantom{.0}& 1187597 & 25400 &  175000   &  2  & 0.2\phantom{0} \\ \hline
\end{tabular} \end{center} }
%
\noindent The plain verification time of the quotient is larger than of the 
original CTMC, despite a state space reduction of a factor two. 
This is due to the fact that the subdominant eigenvalues of the Gauss-Seidel iteration matrices differ significantly---the closer
this value is to one, the slower the convergence rate for the iterative Gauss-Seidel 
method.
For instance for $N = 8$, the values of the original (0.156) and the quotient (0.993)
are far apart and the number of iterations needed differ for about two orders of magnitude.  
The same applies for $N = 16$.  These differences are much smaller for larger values of $N$.

The results for time-bounded reachability:
{\scriptsize
\begin{center}
\begin{tabular}{|r|r|r|r|r|r|r|r|r|r|r|r|r|}
\hline 
%% \hline
%% \multicolumn{11}{|c|}{\emph{Until-formulas for workstation cluster}}\ 
%% \hline \hline
  \multicolumn{5}{|c|}{} & \multicolumn{4}{c|}{time-bounded until $[0,40]$}
  & \multicolumn{4}{c|}{time-bounded until $[20,40]$} \\ \hline
  \multicolumn{5}{|c|}{original CTMC} & \multicolumn{2}{c|}{lumped CTMC}  
  & \multicolumn{2}{c|}{red.\ factor}  & \multicolumn{2}{c|}{lumped CTMC}  
  & \multicolumn{2}{c|}{red.\ factor}\\[1ex] \hline
\multicolumn{1}{|c|}{$N$} & \multicolumn{1}{c|}{states} &transitions&ver.\ time&ver.\ time&blocks& \multicolumn{1}{c|}{time} &states& time& blocks& \multicolumn{1}{c|}{time} &states& time \\
   &         &    & \multicolumn{1}{c|}{$\U^{\leq 40}$} & \multicolumn{1}{c|}{$\U^{[20,40]}$} & &              &      &     &       &                 &     & \\ \hline
  8&    2772 &    12832 &     36 &    49&   239&   16.3          & 11.6 & 2.2 &   386 &   24.0          & 7.2 & 2.0\\
 16&   10132 &    48160 &    360 &   480&   917&   70\phantom{.0}& 11.0 & 5.1 &  1300 &   96.0          & 7.8 & 5.0  \\
 32&   38676 &   186400 &   1860 &  2200&  3599&  300\phantom{.0}& 10.7 & 6.2 &  4742 &  430\phantom{.0}& 8.2 & 5.1 \\
 64&  151060 &   733216 &   7200 &  8500& 14267& 1810\phantom{.0}& 10.6 & 4.0 & 18082 & 2550\phantom{.0}& 8.4 & 3.3 \\
128&  597012 &  2908192 &  29700 & 33700& 56819& 9300\phantom{.0}& 10.5 & 3.2 & 70586 &12800\phantom{.0}& 8.5 & 2.6 \\
256& 2373652 & 11583520 & 121000 &143000&226787&45700\phantom{.0}& 10.5 & 2.6 &278890 &60900\phantom{.0}& 8.5 & 2.3 \\ \hline
\end{tabular} \end{center} }
%
\noindent These results are obtained using a measure-driven bisimulation.
In contrast, for an $\AP$-bisimulation, we only obtained a 50\% state-space reduction.
For measure-driven bisimulation another factor 4--5 reduction is obtained.
	The reduction factors obtained for this case study are not so high,
	as its formal (stochastic Petri net) specification already exploits some lumping;
	e.\,g.\@, workstations are modeled by anonymous tokens.

\paragraph{IEEE 802.11 group communication protocol \cite{MassinkKL_DSN04}.}
This is a variant of the centralized medium access protocol of the IEEE 802.11 standard 
for wireless local area networks.  
The protocol is centralized in the sense that medium access is controlled by a fixed node, the Access Point (AP).  
The AP polls the wireless stations, and on receipt of a poll, stations may broadcast 
a message.  
Stations acknowledge the receipt of a message such that the AP is able to detect whether or
not all stations have correctly received the broadcast message.  
In case of a detected loss, a retransmission by the originator takes place.  
It is assumed that the number of consecutive losses of the same message is bounded by {\it OD}, 
the omission degree.   
This all refers to time-critical messages; other messages are sent in another 
phase of the protocol.  
The property of interest is, as in \cite{MassinkKL_DSN04} and other studies of this protocol, the probability that a message originated by the AP is not received by at
least one station within the duration of the time-critical phase, i.\,e., $t=2.4$ milliseconds, i.\,e.,
$\Prob{\bowtie}{p}{\Diamond^{\leq 24000} \mbox{\it fail}}$ where {\it fail} identifies all states
in which more than {\it OD} losses have taken place. 
The following table reports the results for the verification of this property for different values of 
{\it OD} and the minimization results for a measure-driven bisimulation.

{\scriptsize
\begin{center}
\begin{tabular}{|r|r|r|r|r|r|r|r|}
\hline 
%% \hline
%% \multicolumn{10}{|c|}{\emph{IEEE 802.11 group communication protocol}} \\ 
%% \hline \hline
& \multicolumn{3}{c|}{original CTMC} & \multicolumn{2}{c|}{lumped CTMC} & \multicolumn{2}{c|}{red.\ factor} \\[1ex] \hline 
     {\it OD}            & \multicolumn{1}{c|}{states} &transitions& \multicolumn{1}{c|}{ver.\ time} &blocks& lump + ver.\ time& states       & time \\ \hline
 4    & 1125      & 5369               & 121.9                                & 71         & 13.5      & 15.9    & 9.00     \\
 %% 8    & 9477      & 54953             & 1303\phantom{.0}          & 496      & 129\phantom{.0}      & 19.1    & 10.1   \\
 12   & 37349     & 236313      & 7180\phantom{.0}            & 1821     & 642\phantom{.0}      & 20.5    & 11.2    \\
 %% 16   & 103173    & 686153      & 21248\phantom{.0}        & 4846     & 2145\phantom{.0}   & 21.2    & 9.9     \\
 20   & 231525    & 1590329     & 50133\phantom{.0}       & 10627    & 5431\phantom{.0}  & 21.8    & 9.2     \\
 %% 24   & 453125    & 3183849     & 105328\phantom{.0}     & 20476    & 12294\phantom{.0}  & 22.1    & 8.6     \\
 28   & 804837    & 5750873     & 195086\phantom{.0}     & 35961    & 24716\phantom{.0}  & 22.4    & 7.9     \\
 %% 32   & 1329669   & 9624713     & 324377\phantom{.0}    & 58906    & 45261\phantom{.0}  & 22.6    & 7.2     \\
 36   & 2076773   & 15187833    & 5103900\phantom{.0}   & 91391    & 77694\phantom{.0}  & 22.7    & 6.6     \\
 40   & 3101445   & 22871849    & 7725041\phantom{.0}  & 135752   & 127489\phantom{.0}  & 22.9    & 6.1    \\ \hline
\end{tabular}
\end{center}
}
%
\noindent We obtain a state space reduction of about a factor 22, which results in an efficiency improvement
of a factor 5 to 10.
The reason that the verification times are rather excessive for this model stems from the fact that
the time bound (24000) is very large, resulting in many iterations.
These verification times can be improved by incorporating an on-the-fly steady-state 
detection procedure \cite{KatoenZ_QEST06}, but this is not further considered here. 

\paragraph{Simple P2P protocol \cite{KwiatkowskaNP_CAV06}.}
This case study describes a simple peer-to-peer protocol based on BitTorrent---a ``torrent''
is a small file which contains metadata about the files to be shared and about the host 
computer that coordinates the file distribution. 
The model comprises a set of clients trying to download a file that has been partitioned 
into $K$ blocks. 
Initially, there is a single client that has already obtained all blocks and $N$ additional
clients with no blocks. 
Each client can download a block (lasting an exponential delay) from any of the others 
but they can only attempt four concurrent downloads for each block.
The following table summarises our minimisation results using $\AP$-bisimilarity
in columns 3 through 6.
The property of interest is the probability that all blocks are downloaded within
0.5~time units.
The last columns list the results for a recently proposed symmetry reduction technique 
for probabilistic systems  \cite{KwiatkowskaNP_CAV06} that has been realised in  PRISM.

{\scriptsize
\begin{center}
\begin{tabular}{|c|r|r|r|r|r|r|r|r|r|r|r|r|}
\hline 
%% \hline
%% \multicolumn{11}{|c|}{\emph{Until-formulas for workstation cluster}}\ 
%% \hline \hline
  \multicolumn{3}{|c|}{} & \multicolumn{5}{c|}{bisimulation minimisation}
  & \multicolumn{5}{c|}{symmetry reduction} \\ \hline
  \multicolumn{3}{|c|}{original CTMC} & \multicolumn{3}{c|}{lumped CTMC}  
  & \multicolumn{2}{c|}{red.\ factor}  & \multicolumn{3}{c|}{reduced CTMC}  
  & \multicolumn{2}{c|}{red.\ factor} \\[1ex] \hline 
$N$& \multicolumn{1}{c|}{states} & ver.\ time    &blocks& lump time     & ver.\ time     & \multicolumn{1}{c|}{states} & time       &states&red.\ time& ver.\ time & states & \multicolumn{1}{c|}{time} \\ \hline
2&   1024&    5.6          & 56&    1.4          &   0.3          & 18.3           &3.3 &  528& 12&  2.9          & 1.93          & 0.38 \\
3&  32768&  410\phantom{.0}&252&  170\phantom{.0}&   1.3          & 130\phantom{.0}&2.4 & 5984&100& 59\phantom{.0}& 5.48          & 2.58 \\
4&1048576&22000\phantom{.0}&792&10200\phantom{.0}&   4.8          &1324\phantom{.0}&2.2 &52360&360&820\phantom{.0}&20.0\phantom{0}& 18.3\phantom{0} \\ \hline
\end{tabular}
\end{center}
}
\noindent
We observe that bisimulation minimisation leads to a significantly stronger state-space reduction
than symmetry reduction. 
For $N = 3$ and $N = 4$, bisimulation minimisation leads to a state-space reduction of more than
23 and 66 times, respectively, the reduction of symmetry reduction.
Symmetry reduction is---as expected---much faster than bisimulation minimisation, but this is a
somewhat unfair comparison as the symmetries are indicated manually.
These results suggest that it is affordable to first apply a (fast) symmetry reduction, followed by 
a bisimulation quotienting on the obtained reduced system.
Unfortunately, the available tools did not allow us to test this idea.

\subsection{Rewards}
This section reports on the results for bisimulation minimisation for Markov reward models.
Note that the initial partitions need to be adapted such that only states with
equal reward are grouped. 
We have equipped two DTMCs and one CTMC with a reward assignment function $r$:
%
\begin{itemize}
\item Crowds protocol (DMRM): the reward indicates the number of messages sent;
\item Randomised mutual exclusion protocol (DMRM): the reward indicates the number of attempts 
  that have been undertaken to acquire access to the critical section;
\item Workstation cluster (CMRM): the reward is used to measure the repair time. 
\end{itemize}
%
Recall that for DMRMs, $r(s)$ indicates the reward that is earned on leaving a state, while for CMRMs, 
$r(s){\cdot}t$ is the earned reward when staying $t$ time-units in $s$.
The experiments are focused on verifying time- and reward-bounded until-formulas.
For DMRMs, these formulas are checked using a path graph generation algorithm as proposed
in \cite{AndovaHK_FORMATS03} which has a time complexity in $O(k{\cdot}r{\cdot}|S|^3)$, where $k$ and 
$r$ are the time-bound and reward-bound, respectively.
For CMRMs, we employed the discretization approach by Tijms and Veldman as proposed in 
\cite{HaverkortCHKB_DSN02} which runs in time $O(t{\cdot}r{\cdot}|S|^3{\cdot}d^{{-}2})$ where
$d$ is the step size of the discretisation.
In our experiments, the default setting is $d = \frac{1}{32}$.

For the Crowds protocol (for $R=3$), we checked the probability that the sender's id is discovered
within 100 steps and maximally two messages, i.\,e., $\mathcal{P}_{\leq p}(\Diamond^{\leq 100}_{\leq 2}
\mbox{\it observe})$.
In case of the randomised mutual exclusion protocol, we checked $\Prob{\leq}{q}{\bigwedge_{j \neq 1}^N
\neg \mbox{\sl enter}_j \U^{\leq 50}_{\leq 10} \mbox{\sl  enter}_1}$, i.\,e., maximally 10 attempts
are allowed to enter the critical section.
Finally, for the workstation cluster, we checked the change of providing minimum QoS to premium
QoS within maximally 5 time units of repair (and 10 time units).
All results are listed in the following table.

Due to the prohibitive (practical) time-complexity, manageable state space sizes are (much) smaller than
for the case without rewards.
Another consequence of these large verification times, bisimulation minimisation is relatively cheap,
and results in possibly drastic time savings, as for the Crowds protocol. 
  
{\scriptsize
\begin{center}
\begin{tabular}{|r|r|r|r|r|r|r|r|}
\hline 
\hline
\multicolumn{8}{|c|}{\emph{Crowds protocol with rewards}} \\ 
\hline \hline
& \multicolumn{3}{c|}{original DTMC} & \multicolumn{2}{c|}{lumped DTMC} & \multicolumn{2}{c|}{red.\ factor} \\[1ex] \hline 
  $N$        & \multicolumn{1}{c|}{states} &transitions& ver.\ time    &blocks& lump + ver.\ time& \multicolumn{1}{c|}{states} & \multicolumn{1}{c|}{time} \\ \hline

 5    & 1198      & 2038        & 2928 & 93       & 44.6    & 12.88    & 65.67    \\
%% ---------------------------------------------------------------------------------------------
 10   & 6563      & 15143       & 80394 & 103      & 73.5    & 63.72    & 1094.49  \\
%% ---------------------------------------------------------------------------------------------
 15   & 19228     & 55948       & 1004981 & 103      & 98.7    & 186.68   & 10182.13 \\
%% ---------------------------------------------------------------------------------------------
 20   & 42318     & 148578      & 5174951 & 103      & 161\phantom{.0}     & 410.85   & 32002.61 \\
\hline \hline
\multicolumn{8}{|c|}{\emph{Randomised mutual exclusion protocol with rewards}} \\
\hline \hline
%% & \multicolumn{3}{c|}{original CTMC} & \multicolumn{2}{c|}{lumped CTMC} & \multicolumn{2}{c|}{red.\ factor} \\[1ex] \hline 
%%  $N$        &  states &transitions& ver.\ time    &blocks& lump + ver.\ time& states       & time \\ \hline
 2    & 188       & 455         & 735      & 151      & 616     & 1.25     & 1.19     \\
%%--------------------------------------------------------------------------------------------
 3    & 2368      & 8272        & 60389    & 1123     & 19010   & 2.11     & 3.18     \\
%%---------------------------------------------------------------------------------------------
 4    & 27600     & 123883      & 5446685  & 5224     & 298038  & 5.28     & 18.28    \\
%% ---------------------------------------------------------------------------------------------
 5    & 308800    & 1680086     & $> 10^7$    & 18501    & 3664530 & 16.69    & --   \\
\hline \hline
\multicolumn{8}{|c|}{\emph{Workstation cluster with rewards}} \\ 
\hline \hline
%% & \multicolumn{3}{c|}{original CTMC} & \multicolumn{2}{c|}{lumped CTMC} & \multicolumn{2}{c|}{red.\ factor} \\[1ex] \hline 
%%  $N$        &  states &transitions& ver.\ time    &blocks& lump + ver.\ time& states       & time \\ \hline
%% 1    | 112       | 400         | 58629375    | 62       | 1.47017e+07| 1.81     | 3.99     |
%% ---------------------------------------------------------------------------------------------
 2    & 276       & 1120        & 278708 & 147      & 55448 & 1.88     & 5.03     \\
%% ---------------------------------------------------------------------------------------------
 3    & 512       & 2192        & 849864 & 268      & 151211 & 1.91     & 5.62     \\
%% ---------------------------------------------------------------------------------------------
 4    & 820       & 3616        & 2110095 & 425      & 347324  & 1.93     & 6.08     \\
%% ---------------------------------------------------------------------------------------------
 5    & 1200      & 5392        & $> 10^7$    & 618      & 2086575 & 1.94     & --   \\
%% ---------------------------------------------------------------------------------------------
 6    & 1652      & 7520        & $> 10^7$    & 847      & 3657682 & 1.95     & --   \\
%% ---------------------------------------------------------------------------------------------
\hline
\end{tabular}
\end{center}
}

\section{Concluding remarks}
\label{sec:concl}

Our experiments confirm that significant (up to logarithmic) state space reductions
can be obtained using bisimulation minimisation.
The appealing feature of this abstraction technique is that it is fully automated.
For several case studies, also substantial reductions in time have been obtained
(up to a factor 25).
This contrasts results for traditional model checking where bisimulation minimisation
typically outweighs verifying the original system.
Time reduction strongly depends on the number of transitions in the Markov chain, its 
structure, as well as on the convergence rate of numerical computations.
The P2P protocol experiment shows encouraging results compared with symmetry reduction 
\cite{KwiatkowskaNP_CAV06} (where symmetries are detected manually).
For measure-driven bisimulation for models without rewards, this speedup comes with
no memory penalty: the peak memory use is typically unchanged; for ordinary bisimulation
some experiments showed an increase of peak memory up to 50\,\%.  
In our case studies with rewards, we experienced a 20--40\,\% reduction in peak memory 
use.

We plan to further investigate combinations of symmetry reduction with bisimulation
minimisation, and to extend our experimental work towards MDPs and simulation
preorders.

{\small
\paragraph{Acknowledgement.}
This research has been performed as part of the MC=MC project that is financed by the Netherlands Organization for Scientific Research (NWO), and the project VOSS2 that is financed by NWO and the
German Research Council (DFG).
}

%\input{refs}
%\bibliographystyle{splncs}
{\small
	%Bibliography
	\bibliography{../../../../../Thesis/thesis}
% %\bibliography{refs}
% \begin{thebibliography}{10}
% 
% \bibitem{AndovaHK_FORMATS03}
% Andova, S., Hermanns, H., Katoen, J.-P.:
% \newblock Discrete-time rewards model-checked.
% \newblock In Larsen, K.~G.,  et~al. (eds.): \emph{FORMATS}. \emph{LNCS}, Vol.\@
%   2791.
% \newblock Springer, Berlin (2003)  88--104
% 
% \bibitem{AzizSSB_ACMTCL00}
% Aziz, A., Sanwal, K., Singhal, V., Brayton, R.:
% \newblock Model-checking continuous time {M}arkov chains.
% \newblock \emph{ACM TOCL} \textbf{1} (2000)  162--170
% 
% \bibitem{AzizSSBSV_CAV95}
% Aziz, A., Singhal, V., Balarin, F., Brayton, R.~K., Sangiovanni-Vincentelli,
%   A.~L.:
% \newblock It usually works: the temporal logic of stochastic systems.
% \newblock In Wolper, P. (ed.): \emph{CAV}. \emph{LNCS}, Vol.\@ 939.
% \newblock Springer, Berlin (1995)  155--165
% 
% \bibitem{BaierCG_SIGMETRICS05}
% Baier, C., Ciesinski, F., Gr{\"o}{\ss}er, M.:
% \newblock {P}rob{M}ela and verification of {M}arkov decision processes.
% \newblock \emph{Performance Evaluation Review} \textbf{32} (2005)  22--27
% 
% \bibitem{BaierHHK_ICALP00}
% Baier, C., Haverkort, B., Hermanns, H., Katoen, J.-P.:
% \newblock On the logical characterisation of performability properties.
% \newblock In Montanari, U.,  et~al. (eds.): \emph{ICALP}. \emph{LNCS}, Vol.\@
%   1853.
% \newblock Springer, Berlin (2000)  780--792
% 
% \bibitem{BaierHHK_TSE03}
% Baier, C., Haverkort, B., Hermanns, H., Katoen, J.-P.:
% \newblock Model-checking algorithms for continuous-time {M}arkov chains.
% \newblock \emph{IEEE TSE} \textbf{29} (2003)  524--541
% 
% \bibitem{BaierKHW_IC05}
% Baier, C., Katoen, J.-P., Hermanns, H., Wolf, V.:
% \newblock Comparative branching-time semantics for {M}arkov chains.
% \newblock \emph{Information and Computation} \textbf{200} (2005)  149--214
% 
% \bibitem{MamounPY_EQST06}
% Ben~Mamoun, M., Pekergin, N., Youn{\`e}s, S.:
% \newblock Model checking of continuous-time {M}arkov chains by closed-form
%   bounding distributions.
% \newblock In: \emph{QEST}.
% \newblock IEEE CS, Los Alamitos (2006)  189--198
% 
% \bibitem{BodeHHJPPWB_QEST06}
% B{\"o}de, E., Herbstritt, M., Hermanns, H., Johr, S., Peikenkamp, T., Pulungan,
%   R., Wimmer, R., Becker, B.:
% \newblock Compositional performability evaluation for \textsc{State\-mate}.
% \newblock In: \emph{QEST}.
% \newblock IEEE CS, Los Alamitos (2006)  167--178
% 
% \bibitem{Buchholz_JAP94}
% Buchholz, P.:
% \newblock Exact and ordinary lumpability in finite {M}arkov chains.
% \newblock \emph{Journal of Applied Probability} \textbf{31} (1994)  59--75
% 
% \bibitem{DAprileDS_DS04}
% D'Aprile, D., Donatelli, S., Sproston, J.:
% \newblock {CSL} model checking for the {G}reat{SPN} tool.
% \newblock In Aykanat, C.,  et~al. (eds.): \emph{Computer and Information
%   Sciences, ISCIS}. \emph{LNCS}, Vol.\@ 3280.
% \newblock Springer, Berlin (2004)  543--553
% 
% \bibitem{DArgenioJJL_PAPMPROBMIV01}
% D'Argenio, P.~R., Jeannet, B., Jensen, H.~E., Larsen, K.~G.:
% \newblock Reachability analysis of probabilistic systems by successive
%   refinements.
% \newblock In de~Alfaro, L.,  et~al. (eds.): \emph{PAPM--PROBMIV}. \emph{LNCS},
%   Vol.\@ 2165.
% \newblock Springer, Berlin (2001)  39--56
% 
% \bibitem{Derisavi_PhD05}
% Derisavi, S.:
% \newblock \emph{Solution of {L}arge {M}arkov {M}odels using {L}umping
%   {T}echniques and {S}ymbolic {D}ata {S}tructures}.
% \newblock PhD thesis, Univ.\@ of Illinois at Urbana-Champaign (2005)
% 
% \bibitem{DerisaviHS_IPL03}
% Derisavi, S., Hermanns, H., Sanders, W.~H.:
% \newblock Optimal state-space lumping in {M}arkov chains.
% \newblock \emph{IPL} \textbf{87} (2003)  309--315
% 
% \bibitem{FecherLW_SPIN06}
% Fecher, H., Leucker, M., Wolf, V.:
% \newblock Don't know in probabilistic systems.
% \newblock In Valmari, A. (ed.): \emph{Model Checking Software}. \emph{LNCS},
%   Vol.\@ 3925.
% \newblock Springer, Berlin (2006)  71--88
% 
% \bibitem{FislerV_FMCAD98}
% Fisler, K., Vardi, M.~Y.:
% \newblock Bisimulation minimization in an automata-theoretic verification
%   framework.
% \newblock In Gopalakrishnan, G.,  et~al. (eds.): \emph{FMCAD}. \emph{LNCS},
%   Vol.\@ 1522.
% \newblock Springer, Berlin (1998)  115--132
% 
% \bibitem{FislerV_CHARME99}
% Fisler, K., Vardi, M.~Y.:
% \newblock Bisimulation and model checking.
% \newblock In Pierre, L.,  et~al. (eds.): \emph{CHARME}. \emph{LNCS}, Vol.\@
%   1703.
% \newblock Springer, Berlin (1999)  338--342
% 
% \bibitem{FislerV_FMSD02}
% Fisler, K., Vardi, M.~Y.:
% \newblock Bisimulation minimization and symbolic model checking.
% \newblock \emph{Formal Methods in System Design} \textbf{21} (2002)  39--78
% 
% \bibitem{GroesserB_FMCOB05}
% Groesser, M., Baier, C.:
% \newblock Partial order reduction for {M}arkov decision processes: a survey.
% \newblock In de~Boer, F.~S.,  et~al. (eds.): \emph{FMCO}. \emph{LNCS}, Vol.\@
%   4111.
% \newblock Springer, Berlin (2006)  408--427
% 
% \bibitem{HanssonJ_FAC94}
% Hansson, H., Jonsson, B.:
% \newblock A logic for reasoning about time and reliability.
% \newblock \emph{Formal Aspects of Computing} \textbf{6} (1994)  512--535
% 
% \bibitem{HaverkortCHKB_DSN02}
% Haverkort, B., Cloth, L., Hermanns, H., Katoen, J.-P., Baier, C.:
% \newblock Model checking performability properties.
% \newblock In: \emph{DSN}.
% \newblock IEEE CS, Los Alamitos (2002)  103--112
% 
% \bibitem{HaverkortHK_SRDS00}
% Haverkort, B.~R., Hermanns, H., Katoen, J.-P.:
% \newblock On the use of model checking techniques for quantitative
%   dependability evaluation.
% \newblock In: \emph{19th IEEE Symposium on Reliable Distributed Systems}.
% \newblock IEEE CS, Los Alamitos (2000)  228--237
% 
% \bibitem{HermannsKNPS_JLAP03}
% Hermanns, H., Kwiatkowska, M., Norman, G., Parker, D., Siegle, M.:
% \newblock On the use of {MTBDD}s for performability analysis and verification
%   of stochastic systems.
% \newblock \emph{J.\@ of Logic and Alg.\@ Progr.\@} \textbf{56} (2003)  23--67
% 
% \bibitem{Hillston_96}
% Hillston, J.:
% \newblock \emph{A Compositional Approach to Performance Modelling}.
% \newblock Cambridge University Press (1996)
% 
% \bibitem{Huth_VSS04}
% Huth, M.:
% \newblock An abstraction framework for mixed non-deterministic and
%   probabilistic systems.
% \newblock In Baier, C.,  et~al. (eds.): \emph{Validation of Stochastic
%   Systems}. \emph{LNCS}, Vol.\@ 2925.
% \newblock Springer, Berlin (2004)  419--444
% 
% \bibitem{Huth_TCS05}
% Huth, M.:
% \newblock On finite-state approximants for probabilistic computation tree
%   logic.
% \newblock \emph{TCS} \textbf{346} (2005)  113--134
% 
% \bibitem{IbeT_SAC90}
% Ibe, O.~C., Trivedi, K.~S.:
% \newblock Stochastic {P}etri net models of polling systems.
% \newblock \emph{IEEE J.\ on Selected Areas in Communications} \textbf{8} (1990)
%    1649--1657
% 
% \bibitem{ItaiR_IC90}
% Itai, A., Rodeh, M.:
% \newblock Symmetry breaking in distributed networks.
% \newblock \emph{Information and Computation} \textbf{88} (1990)  60--87
% 
% \bibitem{KatoenKZ_QEST05}
% Katoen, J.-P., Khattri, M., Zapreev, I.~S.:
% \newblock A {M}arkov reward model checker.
% \newblock In: \emph{QEST}.
% \newblock IEEE CS, Los Alamitos (2005)  243--244
% 
% \bibitem{KatoenZ_QEST06}
% Katoen, J.-P., Zapreev, I.~S.:
% \newblock Safe on-the-fly steady-state detection for time-bounded reachability.
% \newblock In: \emph{QEST}.
% \newblock IEEE CS, Los Alamitos (2006)  301--310
% 
% \bibitem{HintonKNP_TACAS06}
% Kwiatkowska, M., Norman, G., Parker, D.:
% \newblock Probabilistic symbolic model checking with {HintonKNP_TACAS06}: a hybrid
%   approach.
% \newblock \emph{Int.\@ J. on STTT} \textbf{6} (2004)  128--142
% 
% \bibitem{KwiatkowskaNP_QEST06}
% Kwiatkowska, M., Norman, G., Parker, D.:
% \newblock Game-based abstraction for {M}arkov decision processes.
% \newblock In: \emph{QEST}.
% \newblock IEEE CS, Los Alamitos (2006)  157--166
% 
% \bibitem{KwiatkowskaNP_CAV06}
% Kwiatkowska, M., Norman, G., Parker, D.:
% \newblock Symmetry reduction for probabilistic model checking.
% \newblock In Ball, T.,  et~al. (eds.): \emph{CAV}. \emph{LNCS}, Vol.\@ 4144.
% \newblock Springer, Berlin (2006)  234--248
% 
% \bibitem{LarsenS_IC91}
% Larsen, K.~G., Skou, A.:
% \newblock Bisimulation through probabilistic testing.
% \newblock \emph{Information and Computation} \textbf{94} (1991)  1--28
% 
% \bibitem{MassinkKL_DSN04}
% Massink, M., Katoen, J.-P., Latella, D.:
% \newblock Model checking dependability attributes of wireless group
%   communication.
% \newblock In: \emph{DSN}.
% \newblock IEEE CS, Los Alamitos (2004)  711--720
% 
% \bibitem{Monniaux_SCP05}
% Monniaux, D.:
% \newblock Abstract interpretation of programs as {M}arkov decision processes.
% \newblock \emph{Science of Computer Programming} \textbf{58} (2005)  179--205
% 
% \bibitem{PnueliZ_DC86}
% Pnueli, A., Zuck, L.:
% \newblock Verification of multiprocess probabilistic protocols.
% \newblock \emph{Distributed Computing} \textbf{1} (1986)  53--72
% 
% \bibitem{ReiterR_ACMTISS98}
% Reiter, M.~K., Rubin, A.~D.:
% \newblock Crowds: anonymity for web transactions.
% \newblock \emph{ACM Transactions on Information and System Security} \textbf{1}
%   (1998)  66--92
% 
% \bibitem{Sproston_TSE06}
% Sproston, J., Donatelli, S.:
% \newblock Backward bisimulation in {M}arkov chain model checking.
% \newblock \emph{IEEE TSE} \textbf{32} (2006)  531--546
% 
% \end{thebibliography}
}

\end{document}
